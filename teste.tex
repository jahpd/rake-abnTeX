%% Template gerado com ruby-abnTeX v0.0.1
\documentclass[12pt, article, a4paper, brazil]{abntex2}\usepackage[T1]{fontenc}		      % seleção de códigos de fonte.
\usepackage[utf8]{inputenc}		      % determina a codificação utiizada (conversão automática dos acentos)
\usepackage[brazil]{babel}	              % idiomas
\usepackage{hyperref}  			      % controla a formação do índice
\usepackage{parskip}
\usepackage[alf]{abntex2cite}                 % Citações padrão ABNT
\usepackage[brazilian,hyperpageref]{backref}  % indique quantas vezes e em quais páginas a citação ocorreu)
\usepackage{draftcopy}                        % Remove this after generate ultimate version of document
\usepackage{blindtext}                        % Remove this after generate with rake

\changes{Vers\~{a}o inicial }{dia: 2014-07-25 22:41:13 -0300 }{v0.0.1}
\titulo{Artigo gerado teste com rake-\abnTeX} \thanks{\imprimirtipotrabalho desenvolvido com \LaTeX; formatado com \abnTeX}}
\autor{Autor \url{autor@mail.com} \thanks{Alguma nota sobre o autor e a pesquisa}}
\instituicao{Universidade Publica -- UP
  \par 
  Faculdade Publica -- FP
  \par
  Programa de Estudos
}
\tipotrabalho{Trabalho academico}
\orientador[Orientador]{Prof. Dr. Orientador}
\coorientador[Co-Orientador]{Prof. Dr. Co-Orientador}
\data{\today}\makeatletter
\hypersetup{
  pdftitle={\@title},
  pdfauthor={\@author},
  pdfsubject={\@imprimirpreambulo},
  pdfkeywords={palavra}{chave},
  pdfcreator={\LaTeX with \abnTeX2},
  colorlinks=true,
  linkcolor=blue,
  citecolor=blue,
  urlcolor=blue
}
\makeatother

\begin{document}
\maketitle

\imprimirinstituicao

\imprimirorientadorRotulo \imprimirorientador

\imprimircoorientadorRotulo \imprimircoorientador

\begin{abstract}
  \blindtext
\end{abstract}

\section{Introducao}
  Escreva aqui algo sobre Introducao \cite{exemplo_artigo_2014}; voc\^e pode usar comandos \LaTeX e \abnTeX \cite{abntex_tutorial_2014}} com a dupla barra \Blindtext 
  \Blindtext  %% Remova isso!!!!

\section{Exemplos}
  Escreva aqui algo sobre Exemplos \cite{exemplo_artigo_2014}; voc\^e pode usar comandos \LaTeX e \abnTeX \cite{abntex_tutorial_2014}} com a dupla barra \Blindtext 
  \Blindtext  %% Remova isso!!!!

\section{Conclusao}
  Escreva aqui algo sobre Conclusao \cite{exemplo_artigo_2014}; voc\^e pode usar comandos \LaTeX e \abnTeX \cite{abntex_tutorial_2014}} com a dupla barra \Blindtext 
  \Blindtext  %% Remova isso!!!!

\bibliography{teste}
\end{document}
